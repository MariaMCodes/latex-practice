\documentclass[12pt]{article}

\usepackage{amsmath}

\usepackage{hyperref}

\hypersetup{colorlinks=true, linkcolor=blue, urlcolor=blue, citecolor=blue}
\usepackage{graphicx}

\title{Hello World!}

\author{Maria Mempin}

\date{January 1, 1831}

\begin{document}
\maketitle    
\section{Getting Started}
\textbf{Hello World!} Today I am learning \LaTeX. \LaTeX{} is a great program for writing math.  I can write in line math such as $a^2 + b^2 = c^2$.   I can also give equations their own space: 
\begin{equation}
    \gamma^2 +\theta^2 = \omega^2
\end{equation}
``Maxwell's equations" are named for James Clark Maxwell and are as follow:
\begin{align}
    \vec{\nabla} \cdot \vec{E} \quad &=\quad\frac{\rho}{\epsilon_0} &&\text{Gauss's Law} \label{eq:GL}\\      
    \vec{\nabla} \cdot \vec{B} \quad &=\quad 0 &&\text{Gauss's Law for Magnetism} \label{eq:GLM}\\
    \vec{\nabla} \times \vec{E} \quad &=\hspace{10pt}-\frac{\partial{\vec{B}}}{\partial{t}} &&\text{Faraday's Law of Induction} \label{eq:FL}\\ 
    \vec{\nabla} \times \vec{B} \quad &=\quad \mu_0\left( \epsilon_0\frac{\partial{\vec{E}}}{\partial{t}}+\vec{J}\right) &&\text{Ampere's Circuital Law} \label{eq:ACL}
    \end{align}
    Equations \ref{eq:GL}, \ref{eq:GLM}, \ref{eq:FL}, and \ref{eq:ACL} are some of the most important in Physics.
    \section{What about Matrix Equations?}
    \begin{equation*}
        \begin{pmatrix}
        a_{11}&a_{12}&\dots&a_{1n}\\
        a_{21}&a_{22}&\dots&a_{2n}\\
        \vdots&\vdots&\ddots&\vdots\\
        a_{n1}&a_{n2}&\dots&a_{nn}
        \end{pmatrix}
        \begin{bmatrix}
        v_{1}\\
        v_{2}\\
        \vdots\\
        v_{n}
        \end{bmatrix}
        =
        \begin{matrix}
        w_{1}\\
        w_{2}\\
        \vdots\\
        w_{n}
        \end{matrix}
        \end{equation*}
    \section{Tables and Figures}
Creating a Table is not unlike creating a matrix:
\begin{table}[h!]
    \centering
    \caption{This is a table that shows how to create different lines as well as different justifications}
    \begin{tabular}{|l||c|c|r|}
    \hline
    $x$&1&2&3\\
    \hline
    $f(x)$&4&8&12\\
    f(x)&4&8&12\\
    \hline
    \end{tabular}
\end{table}

\begin{figure}[h!]
    \centering
    \includegraphics[width=\textwidth]{bg-8}
\caption{Bern Dibner Library}
\end{figure}
\section{References}
You will probably want references in your document so that you can cite articles like \cite{frenkel_fine_2013, frenkel_optical_2013, frenkel_temperature_2012, frenkel_whispering-gallery_2013,frenkel_-chip_2016}

\bibliography{bibl}
\end{document}
